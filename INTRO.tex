\chapter*{Introdução}
\addcontentsline{toc}{chapter}{Introdução, \emph{por Paul D. Escott}}
\hedramarkboth{Introdução}{}

\begin{flushright}
\versal{PAUL D. ESCOTT}\\\vspace{-3pt}
\versal{WAKE FOREST UNIVERSITY}
\end{flushright}

Como podemos conhecer os escravos? Como aprender sobre os homens,
mulheres e crianças escravizados nos Estados Unidos antes da Guerra
Civil? O que podemos fazer para entender suas ideias, sentimentos,
esperanças e desejos e saber como foi a experiência da escravidão?

Em 1860, às vésperas da Guerra Civil, que levou à abolição da
escravatura nos \versal{EUA} sob o presidente Abraham Lincoln, havia quase 4
milhões de escravos nos estados sulistas. Eles eram uma parte importante
da sociedade do Sul, tanto nos estados mais antigos no Leste, como a
Virgínia, as Carolinas ou a Geórgia, quanto nas áreas em franca expansão
no Oeste, como a Luisiana ou o Texas. O seu trabalho produzia as
riquezas imensas do cultivo de algodão, além de outras culturas
comerciais, como açúcar, arroz e tabaco. Por si só, o algodão
representava metade do valor de todas as exportações americanas nas três
décadas anteriores à Guerra Civil. O ``valor monetário'' dos escravos
era maior do que a soma de todos os investimentos do país em ferrovias e
manufaturas, e por causa da mão de obra negra, os grandes senhores de
escravos eram 70\% das pessoas mais ricas dos \versal{EUA}.\footnote{Média
  calculada a partir dos Relatórios do Secretário do Tesouro; James L.
  Huston, \emph{Calculating the Value of the Union: Slavery, Property
  Rights, and the Economic Origins of the Civil War} (Chapel Hill: The
  University of North Carolina Press, 2003, p.~25, 27-30).} Contudo, as
vidas dos escravos não são bem documentadas. As leis estaduais e
práticas sociais rígidas proibiam que eles aprendessem a ler e a
escrever. Os escravos eram uma população oprimida e vigiada de perto, e
apesar de alguns indivíduos terem conseguido se alfabetizar ou fugir do
cativeiro, temos poucas fontes primárias nas suas próprias palavras. Os
proprietários de escravos e os brancos do sul nos deixaram uma imensa
quantidade de cartas, diários, documentos comerciais e jornais, mas os
escravos não tinham permissão para se expressarem da mesma forma.

O historiador não pode confiar nas opiniões e descrições dos senhores
como únicas fontes de informações sobre os escravos. É por isso que as
narrativas de escravos são de suma importância. Há dois conjuntos de
narrativas: aquelas publicadas no século \versal{XIX}, geralmente em torno da
Guerra Civil, e um número maior de narrativas coletadas posteriormente,
no início do século \versal{XX}. Durante o \versal{XIX}, alguns homens e mulheres
conseguiram escapar da escravidão e publicaram suas memórias, muitas
vezes com a ajuda de abolicionistas do norte.\footnote{Frederick
  Douglass, \emph{Narrative of the Life of Frederick Douglass, an
  American Slave, Written by Himself}, edited with an Introduction by
  David W. Blight, Third edition (Boston and New York: Bedford/St.
  Martin's, 2017, 2003, 1993); Harriet Jacobs, \emph{Incidents in the
  Life of a Slave Girl} (Black \& White Classics, 1861 Edition, New
  York, copyright 2014). Ver também John W. Blassingame, editor,
  \emph{Slave Testimony} (Baton Rouge: Louisiana State University Press,
  1977). As narrativas de Douglass, Brown e Jacobs estão sendo
  publicadas como parte desta coleção.} Algumas obras, escritas por
indivíduos como Frederick Douglass, William Wells Brown e Harriet
Jacobs, ajudaram a fortalecer o sentimento antiescravista nos Estados
Unidos e se tornaram famosas. Mas as narrativas do século \versal{XX} representam
um recurso muito maior, principalmente aquelas coletadas pelo Projeto
Federal de Escritores (\versal{FWP}, Federal Writers' Project) durante a
presidência de Franklin Delano Roosevelt (1933-1945). Essas narrativas
do \versal{FWP} são o foco deste volume.

\section{As Narrativas de Escravos do Projeto Federal de Escritores e Suas
Características}

Na década de 1920, alguns pesquisadores da University of Tennessee, Fisk
University e Southern University de Luisiana começaram a coletar as
memórias de alguns afro"-americanos mais velhos que haviam vivenciado a
escravidão. Cientes que a morte reduzia o número de ex"-escravos todos os
anos, esses pesquisadores começaram a buscar informações daqueles que
haviam vivido sob essa instituição. Seus projetos eram pequenos, mas
demonstraram o valor de buscar informações junto aos ex"-escravos. Com a
chegada da Grande Depressão e uma crise econômica grave de vários anos,
a administração do presidente Franklin Roosevelt tentou reavivar a
economia e socorrer os mais necessitados. O governo federal lançou uma
ampla variedade de programas para reformar o sistema econômico e ajudar
os desempregados. Escritores profissionais e aspirantes participaram do
Projeto Federal de Escritores, que realizou uma série de projetos
diferentes, desde guias dos diferentes estados até entrevistas com
ex"-escravos idosos. Entre 1936 e 1938, entrevistadores em dezessete
estados resumiram suas conversas com ex"-escravos e mandaram quase 2.400
narrativas datilografadas para Washington, \versal{DC}.

No início, a academia ignorou essas fontes. A Guerra Civil dera fim à
escravidão legal, mas não estabelecera direitos e oportunidades iguais
para os ex"-escravos. Na década imediatamente após o conflito, os homens
negros conquistaram o direito ao voto, mas os governos estaduais que
simpatizavam com os seus interesses logo foram derrotados no Sul. Na
última década do século \versal{XIX}, governos reacionários em todos os estados
sulistas aprovaram leis que praticamente roubavam o voto dos
afro"-americanos. Outras leis impuseram um sistema de discriminação
humilhante em locais públicos e serviços governamentais, um sistema de
discriminação que conquistou o selo de aprovação da Suprema Corte dos
Estados Unidos em 1896.\footnote{A decisão da Suprema Corte é conhecida
  pelo nome \emph{Plessy v. Ferguson}.} Assim, no início do século \versal{XX}, a
segregação racial era uma força dominante na sociedade sulista e o
preconceito racial era forte e disseminado no resto do país.

Os mais influentes entre os primeiros estudiosos da escravidão, como U.
B. Phillips, em geral compartilhavam das ideias da supremacia branca.
Phillips escreveu que a escravidão fora uma escola para os africanos
incivilizados e rejeitou as narrativas de escravos, dizendo que não eram
confiáveis e não tinham valor. Foi só nas décadas de 1950 e 1960 que as
atitudes começaram a mudar significativamente, e então o Movimento dos
Direitos Civis despertou as consciências de muitos brancos à medida que
desmantelava as leis segregacionistas. Os historiadores agora ansiavam
para entender melhor o passado racista dos Estados Unidos. Muitos
voltaram a sua atenção para a escravidão e começaram a usar as
Narrativas de Escravos do \versal{FWP}. Em 1972, a Greenwood Publishing Company
publicou uma edição em 19 volumes que reproduzia as transcrições
datilografadas originais.\footnote{Federal Writers' Project, \emph{The
  American Slave: A Composite Autobiography}, George Rawick, General
  Editor (Westport, \versal{CT}: Greenwood Publishing Company, 1972). Greenwood
  também publicou o \emph{Supplement: Series \versal{I}} em 1977 e o
  \emph{Supplement: Series \versal{II}} em 1979.} Posteriormente, foram
publicados volumes adicionais, incluindo trabalhos do Projeto Federal de
Escritores que nunca haviam sido enviados para Washington. Nos últimos
anos, em resposta à popularidade da internet e às vantagens da
tecnologia digital, a Biblioteca do Congresso dos \versal{EUA} colocou as
Narrativas de Escravos do \versal{FWP} na rede. Graças à Biblioteca do Congresso,
hoje qualquer um tem acesso a um pouco da história de vida dos
escravos.\footnote{Disponíveis em:
  \textless{}\emph{https://bit.ly/2WuE7vr}\textgreater{}.}

Como abordar as Narrativas de Escravos do \versal{FWP}? Alguns questionamentos ou
possíveis objeções a elas, como a sua representatividade, não podem ser
ignorados, e as circunstâncias nas quais foram coletadas nos deixam com
um desafio. Tanto historiadores quanto leitores leigos precisam levar
essas questões em conta. O quanto os ex"-escravos foram francos e
honestos? O quanto lembravam sobre a escravidão? As entrevistas que
temos oferecem um retrato útil dos diferentes ambientes e experiências
de quem esteve em cativeiro?

Henry Alsberg, o diretor nacional do Projeto Federal de Escritores,
insistiu com todos os entrevistadores que estes deviam tomar o máximo de
cuidado para não influenciar o ponto de vista do informante e enfatizou
que \emph{todas as histórias deveriam ser reproduzidas palavra por
palavra sempre que possível}. Infelizmente, as suas sugestões não tinham
como garantir que as narrativas seriam uma expressão imaculada, direta e
sem enfeites das perspectivas dos ex"-escravos. Além de seus conselhos
somente terem sido recebidos após os entrevistadores já terem começado a
se encontrar com ex"-escravos em diversos estados, e mais importante
ainda, as circunstâncias sociais da época exigiam que os ex"-escravos
aplicassem nível considerável de cautela ao falarem com eles.\footnote{Para
  mais detalhes sobre as características das entrevistas do \versal{FWP} e
  questões discutidas nesta Introdução, consulte Paul D. Escott,
  \emph{Slavery Remembered: A Record of Twentieth"-Century Slave
  Narratives} (Chapel Hill: The University of North Carolina Press,
  1978).}

A supremacia branca era uma realidade tirânica no Sul da década de 1930.
A subordinação rígida dos negros era a regra no Sul segregado, e isso
naturalmente moldou o jeito como os afro"-americanos interagiam com
pessoas brancas. Se ofendiam um branco, nada, nem mesmo a idade avançada
dos ex"-escravos, poderia protegê"-los de consequências desagradáveis.
Como quase todos os entrevistadores do \versal{FWP} eram brancos, os ex"-escravos
estavam cientes da necessidade de observar todas as regras da etiqueta
racial. O fato dos entrevistadores se apresentarem como agentes do
governo federal também afetou as conversas. Em geral, os ex"-escravos
eram pobres, lutando contra a fome e a pobreza, e tinham a esperança de
obter uma pensão ou alguma outra forma de auxílio do governo. Por
consequência, tanto os ex"-escravos quanto a equipe do \versal{FWP} agiram de
maneiras que afetaram a natureza das narrativas.

Para não ofender os entrevistadores brancos, os ex"-escravos geralmente
começavam dizendo coisas positivas sobre seus donos e a sua experiência
sob a escravidão. Para evitar conflitos, a primeira prioridade era
mostrar à pessoa branca mais poderosa que eles entendiam o seu lugar no
mundo e reconheciam o fato e as regras da subordinação racial.
Obviamente, alguns desses comentários positivos devem ter sido sinceros.
Alguns indivíduos que conviveram intimamente com os seus donos por anos
a fio desenvolviam alguma afeição por essa pessoa, assim como alguns
escravistas religiosos ou piedosos aprenderam a reconhecer a humanidade
daqueles de quem eram os proprietários legais. As condições de extrema
necessidade de outros ex"-escravos também contribuíram para as afirmações
positivas. Quem passava fome ou havia enfrentado anos de miséria
lembrava da escravidão como uma época em que, pelo menos, tinham comida
o suficiente para forrar seus estômagos. Aqueles que foram crianças
durante a escravidão tendiam a lembrar de mais épocas boas e de não
terem sofrido com trabalho árduo ou tratamentos cruéis. Alguns
ex"-escravos podem ter relutado em revelar para um estranho as
experiências humilhantes ou degradantes que sofreram. Mas os comentários
positivos quase nunca eram toda a história, e muitas entrevistas
revelaram casos aterradores.

Pouquíssimos indivíduos tiveram a coragem de condenar a crueldade ou a
injustiça desde o primeiro momento das suas entrevistas. A maioria era
muito mais cautelosa e resguardada do que isso, especialmente entre os
brancos. Como disse um entrevistado: ``não vou contar nada para os
brancos, tenho medo de fazer inimigos''. Outro homem explicou que
``muitos escravos velhos fecham a porta antes de contar a verdade sobre
a época da escravidão. Quando a porta está aberta, eles contam como os
seus senhores eram bonzinhos e como tudo era uma maravilha''. A análise
quantitativa das entrevistas do \versal{FWP} revela que os entrevistadores negros
tinham maior probabilidade de ouvir histórias sobre castigos físicos,
sexo forçado nas fazendas, miscigenação na família do ex"-escravo,
atitudes hostis em relação ao senhor e aspectos da cultura negra, como a
crença em conjuros.

Assim, era apenas nas fases posteriores de uma entrevista que muitos
ex"-escravos começavam a tocar em eventos dolorosos que haviam afetado
eles, suas famílias ou outros escravos da fazenda. Os leitores do
conjunto total das narrativas precisam manter isso em mente quando
procuram as opiniões dos escravos sobre os seus donos. Suas avaliações
são comparativas, partindo de visões sobre um contexto social baseado em
coerção e falta de liberdade. Assim, um comentário típico de um
ex"-escravo é que o seu dono era bom porque não açoitava muito ou era
melhor do que os vizinhos cruéis. Os entrevistadores negros ouviram uma
descrição mais sombria das realidades do cativeiro do que os
trabalhadores brancos do \versal{FWP}.\footnote{Para mais informação sobre essa
  questão e outros temas discutidos nos parágrafos a seguir, consulte a
  Introdução e outros capítulos de \emph{Slavery Remembered}.}

A idade dos entrevistados naturalmente impactou o que eles conseguiam
saber e lembrar sobre a escravidão. Por si só, a idade não destrói o
valor das lembranças do ex"-escravos. A memória humana não é exata, mas a
idade a prejudica muito menos do que os problemas de saúde, e os
participantes do projeto do \versal{FWP} já haviam provado a sua vitalidade,
tendo sobrevivido a muitos da sua geração. Mas os ex"-escravos mais
jovens, aqueles com menos de oitenta anos, tinham sido crianças pequenas
na época da escravidão; quase um quinto deles tinha menos de cinco anos
de idade em 1865. Assim, não é surpresa que as experiências pessoais de
que se recordam tendem a ser mais positivas, com outras lembranças
provavelmente vindo de pais ou outros familiares. Ainda assim, pouco
mais de um terço dos ex"-escravos havia nascido antes de 1851. Esses
indivíduos mais idosos teriam tido idade suficiente para trabalhar no
campo e enfrentar as mesmas dificuldades que os adultos sob a
escravidão.

Quem analisa as narrativas do \versal{FWP} pela primeira vez também nota que um
grande número de entrevistas, quase setecentas delas, vieram do estado
do Arkansas. Na época da Guerra Civil, o Arkansas ainda eram um estado
pouco populoso, no extremo Oeste da expansão da fronteira americana.
Logo, à primeira vista, pode parecer que essas entrevistas não
refletiriam a natureza da escravidão sulista em geral, reduzindo o valor
da coleção da \versal{FWP}. Contudo, muitos escravos abandonaram as suas fazendas
quando conquistaram a liberdade e se mudaram para o Oeste nos anos
subsequentes, estabelecendo"-se em estados como o Texas ou o Arkansas.
Quando examinamos onde os entrevistados haviam vivenciado a escravidão,
esse problema praticamente desaparece. Os estados algodoeiros mais
produtivos estão bem representados nas narrativas. Os estados pequenos e
aqueles nos limites do Sul não dominam o conjunto.

Quando se analisa outros possíveis problemas, parecem surgir mais
motivos para não descontar o valor destas fontes primárias. A análise
quantitativa da coleção do \versal{FWP} mostra que a condição econômica dos
entrevistados em pouco impactava as suas opiniões. Ser dono de imóveis
não afetava as atitudes expressas nas narrativas, e a maioria dos
ex"-escravos era pobre. Os escravos domésticos e as crianças que tinham
relativamente poucas obrigações estão super"-representados nas
entrevistas, mas elas também incluem um número considerável de
ex"-escravos que trabalharam como lavradores. Além disso, a imagem da
escravidão formada pelas narrativas geralmente é crítica, tenha o
ex"-escravo trabalhado na lavoura ou em casa. Para essas e outras
preocupações do tipo, a opinião de C. Vann Woodward, um dos mais
respeitados historiadores americanos do Sul no século \versal{XX}, é apropriada.
``Deve ser evidente'', Woodward escreveu, ``que essas entrevistas com
ex"-escravos precisarão ser utilizadas com cautela e discernimento.
(\ldots{}) Contudo, as precauções necessárias não são mais complexas ou
onerosas do que aquelas exigidas por muitos outros tipos de fonte {[}que
o historiador{]} está acostumado a utilizar''.\footnote{Citado em
  \emph{Slavery Remembered}, página 17.}

\section{As Narrativas de Escravos e Interpretações~da~Escravidão}

Assim, as atitudes dos historiadores em relação às Narrativas de
Escravos do \versal{FWP} mudaram junto com a sociedade americana, e também com as
conquistas do Movimento dos Direitos Civis, os estudiosos adotaram o
ponto de vista do professor Woodward. Em 1972, o professor John
Blassingame publicou \emph{The Slave Community} {[}A Comunidade
Escrava{]}, um livro que soube usar muito bem as narrativas de escravos
do século \versal{XIX}. \emph{The Slave Community} defendeu o argumento
importante de que os escravos dependiam uns dos outros e formavam uma
comunidade e uma cultura nas senzalas, o que permitia que resistissem às
pressões desumanizantes da escravidão. Dois anos depois, o professor
Eugene Genovese publicou \emph{Roll, Jordan, Roll} {[}Corre, Jordão,
Corre{]}, um longo e emocionante volume que faz amplo uso das Narrativas
de Escravos do \versal{FWP}. Citações de fontes do \versal{FWP} ilustram os argumentos de
Genovese ao longo do livro, o que chamou bastante atenção e foi bastante
influente. As grandes obras posteriores sobre a escravidão, incluindo
algumas que foram controversas ou largamente rejeitadas (como \emph{Time
on the Cross} {[}Tempo na Cruz{]}, de Robert Fogel e Stanley Engerman)
utilizaram as narrativas de escravos. Hoje, ninguém duvida que essas
fontes são aceitas e consideradas valiosas e que nenhum estudioso
americano tentaria publicar uma obra de grande porte sobre escravidão
sem recorrer a elas de alguma forma.\footnote{John W. Blassingame,
  \emph{The Slave Community: Plantation Life in the Antebellum South}
  (New York: Oxford University Press, 1972); Eugene D. Genovese,
  \emph{Roll, Jordan, Roll: The World the Slaves Made} (New York:
  Pantheon, 1974); Robert William Fogel and Stanley L. Engerman,
  \emph{Time on the Cross: The Economics of American Negro Slavery}
  (Boston: Little, Brown, 1974).}

Isso não significa, é claro, que todas as questões interpretativas foram
resolvidas ou que todos os historiadores concordam em seu entendimento
da escravidão e da experiência dos escravos no sul dos Estados Unidos.
Alguns pontos controversos nas décadas anteriores atingiram algum nível
de consenso entre os estudiosos. Por exemplo, uma perspectiva que antes
era bastante disseminada, mas hoje é rejeitada, afirmava que os escravos
eram indefesos e incapazes de resistir. Essas ideias se originaram entre
os escravocratas, com proponentes da escravidão e entre os defensores do
Sul no pós"-guerra. Essas pessoas haviam insistido que os escravos eram
criaturas dóceis, infantis e racialmente inferiores, que haviam se
beneficiado dos comandos autoritários ou da tutela dos seus donos. Na
década de 1950, um estudioso argumentou que os escravos eram dóceis
porque a opressão brutal da escravidão os ``infantilizara'', semelhante
ao que pode ter acontecido com alguns sobreviventes dos campos de
concentração nazistas.\footnote{Stanley M. Elkins, \emph{Slavery: A
  Problem in American Institutional and Intellectual Life} (Chicago: The
  University of Chicago Press, 1959). Elkins tomou o conceito de
  infantilização emprestado da obra de Bruno Bettleheim, sobre os campos
  de concentração nazistas.}

A ênfase da década de 1970 em uma cultura escrava alternativa de
resistência, contrária a essas ideias, passou a dominar. Os estudiosos
hoje reconhecem que os escravos faziam praticamente tudo que podiam para
solapar ou resistir ao poder dos seus donos e para construir para si um
conjunto de valores opostos. Eles concordam que as realidades práticas
dominantes explicam por que as grandes revoltas foram relativamente tão
raras nos Estados Unidos em comparação com partes do Caribe ou outras
regiões no hemisfério ocidental. A proporção entre brancos e negros era
muito maior nos Estados Unidos, a população escrava era mais dispersa e
havia menos homens jovens e solteiros e africanos recém"-importados após
a lei de 1808 que proibia a participação no comércio internacional de
escravos. A geografia também jogava contra a resistência armada, pois os
estados sulistas tinham menos áreas onde comunidades quilombolas podiam
se refugiar e expandir. Hoje, ninguém discorda que os escravos do Sul
tinham uma cultura que permitiu que resistissem à desumanização. Alguns
estudiosos discordam sobre diversos elementos dessa cultura ou sobre a
importância da religião nela, mas a comunidade escrava é vista como
resistente e acostumada a disfarçar suas crenças e enganar os brancos.

Outra área em que o consenso se fortaleceu drasticamente nos últimos
anos trata da riqueza criada pela mão de obra escrava e a suma
importância da escravidão para o desenvolvimento econômico dos Estados
Unidos e, mais do que isso, de todo o Atlântico. Até a década de 1860, a
indústria têxtil britânica crescia rapidamente e a demanda mundial por
algodão aumentava a cerca de 5\% ao ano. Após a invenção do descaroçador
de algodão, que removia facilmente as sementes do algodão de fibra
curta, o Sul estava posicionado para se transformar no líder mundial da
produção de algodão cru. A escravidão e o cultivo de algodão se
expandiram rapidamente para o Oeste, atravessando a cordilheira dos
Apalaches e chegando às terras férteis em torno do Golfo do México. Além
da instituição da escravidão se arraigar no Sul, a economia do algodão
criou fortunas enormes para os escravistas e alimentou outros negócios e
empreendimentos mercantis no país. A riqueza produzida pelo cultivo de
algodão logo superou aquela criada nas economias açucareiras do Caribe e
provocou mudanças econômicas e industriais em nível mundial.\footnote{Ver,
  por exemplo, obras recentes como: Sven Beckert, \emph{Empire of
  Cotton: A Global History} (New York: Knopf, 2014); Edward E. Baptist,
  \emph{The Half Has Never Been Told: Slavery and the Making of America
  Capitalism} (New York: Basic Books, 2014); Walter Johnson, \emph{River
  of Dark Dreams: Slavery and Empire in the Cotton Kingdom} (Cambridge,
  \versal{MA}: Belknap Press, 2013). \emph{The Political Economy of the Cotton
  South: Households, Markets, and Wealth in the Nineteenth Century} (New
  York: Norton, 1978), de Gavin Wright, continua a ser um marco
  essencial para entender o Sul algodoeiro.}

Ainda há controvérsias significativas sobre qual seria a melhor maneira
de categorizar ou analisar a economia algodoeira do Sul. Ela era
totalmente integrada à rede capitalista e comercial global, mas em
alguns aspectos importantes, também era menos desenvolvida e menos
capitalista do que outras regiões dos Estados Unidos e da Inglaterra.
Partindo de uma perspectiva interpretativa marxista, o argumento
influente de Eugene Genovese é que a sociedade escravocrata sulista era
senhoril ou pré"-capitalista. Genovese insistia que os escravistas
possuíam um conjunto de valores que os diferenciava dos capitalistas e
do domínio do nexo monetário. Contudo, outros autores rejeitam esse
ponto de vista. Eles identificaram líderes sulistas cujo discurso era
favorável à indústria ou apresentam evidências de que o Sul, apesar de
ter começado mais tarde, estava construindo ferrovias ao mesmo ritmo que
o resto dos Estados Unidos. A controvérsia sobre a natureza da economia
sulista e se ela era ou não capitalista é um debate absolutamente
presente e promete continuar no futuro.

Outra questão interpretativa importante enfoca a relação entre senhores
e escravos nos Estados Unidos. Mais uma vez, a obra de Eugene Genovese
está no centro da controvérsia. Genovese admirava e utilizava o marco
teórico de Antonio Gramsci, segundo o qual as classes dominantes mantêm
sua hegemonia sobre os dominados por meio da construção do conjunto de
ideias nos quais estes estão inseridos. Ao adaptar as leis e os
princípios sociais aos seus interesses, os escravistas estabeleciam um
terreno seguro para resolver apenas conflitos que não colocavam em xeque
o seu poder. Para Genovese, o mecanismo que deu aos escravocratas o
poder hegemônico sobre os seus escravos foi o paternalismo, a relação
que os donos promoviam com os seus escravos nas fazendas. Os fazendeiros
descreviam essa relação como amorosa, com o proprietário assumindo o
papel de pai. Genovese insistia que a essência desse paternalismo era de
exploração, não de carinho e proteção, mas também argumentava que os
laços pessoais entre os livres e os escravizados prendiam os escravos
mentalmente aos seus donos e à sua fazenda, e não a uma classe mais
ampla que poderia contemplar uma revolta. O paternalismo que Genovese
enxergava na escravidão delimitava e restringia a visão de mundo dos
afro"-americanos.

Outros autores discordam, grupo no qual me incluo. Em termos econômicos,
os escravos eram uma classe oprimida, mas a realidade social que
vivenciavam todos os dias os lembrava todos os dias que também eram uma %%todos os dias²
raça desprezada. O racismo fanático foi, afinal, uma característica
marcante da escravidão nos Estados Unidos. Apesar do fato da
miscigenação, a sociedade não reconhecia a multiplicidade de grupos
raciais. Em vez disso, os Estados Unidos seguiam uma regra de ``uma
gota'', na qual a pessoa que tinha um único ancestral africano era
classificada como negra e, logo, consignada a um status social inferior.
Os escravos encontravam o racismo branco no seu cotidiano e eram
informados constantemente que os ``crioulos'' eram inferiores. Além
disso, a escravidão se baseava em coerção, em castigos físicos aplicados
sempre que um escravo não fazia o que o seu dono queria. Apenas os
negros eram escravos, e apenas os escravos negros recebiam esse tipo de
tratamento. Creio que essas experiências aprofundaram naturalmente o seu
conceito de identidade racial, aguçaram a sua consciência sobre a
exploração branca e limitaram o efeito que o paternalismo poderia ter na
sua mentalidade. Aqueles que enfatizam o papel do racismo, de encontro a
Genovese, dão muito menos importância ao paternalismo e enxergam uma
distância muito maior entre senhores e escravos.

Controvérsias como essas são normais nas investigações históricas sobre
questões importantes e é improvável que a disponibilidade das Narrativas
de Escravos do \versal{FWP} vá resolvê"-las. Mas é certo que sem essa grande
coleção das memórias de ex"-escravos, teríamos muito menos material para
trabalharmos e nossas análises da experiência sob a escravidão seriam
mais fracas e mais tênues. Para os estudiosos e também para os leitores
interessados, as Narrativas de Escravos do \versal{FWP} oferecem uma perspectiva
valiosa, praticamente única, sobre como eram as vidas dos escravos no
sul dos Estados Unidos.

\section{O Que Esperar Deste Livro}

O restante deste volume apresenta passagens extraídas das narrativas de
escravos, organizadas em oito seções ou capítulos. Cada capítulo enfoca
um tema que esclarece um aspecto importante da escravidão. Os oito são:
Trabalho, Condições de vida, Crueldade e castigos físicos, Famílias,
Atitudes raciais, Cultura negra, Resistência e Emancipação.

A extensão e o tom das entrevistas com ex"-escravos teve algum nível de
variação, dependendo das personalidades dos entrevistadores e dos
entrevistados, mas para dar ao leitor uma ideia sobre como seria uma
entrevista completa, a maioria dos temas é introduzida com o texto
completo de uma entrevista razoavelmente típica com um ex"-escravo. Como
veremos, muitas das passagens reunidas aqui não se concentram
exclusivamente no tema do capítulo. Como os entrevistadores faziam
perguntas sobre uma longa lista de tópicos e depois resumiam tudo que
fora dito, os comentários dos ex"-escravos podem abranger uma ampla
variedade de temas.

Dentro de cada seção, à medida que as passagens oferecem mais detalhes
sobre o tema, adicionei comentários onde acreditei que seriam úteis para
explicar o contexto das declarações de um ex"-escravo ou para fornecer
informações úteis sobre as realidades econômicas e sociais e as
variações internas da escravidão nos \versal{EUA}. Para distingui"-los das
palavras dos ex"-escravos, todos esses comentários adicionais aparecem em
fonte menor e sem recuo.